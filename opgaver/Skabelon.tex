\documentclass[10pt,a4paper,danish]{article}
\usepackage[danish]{babel}
\usepackage[utf8]{inputenc}
\usepackage{amsmath}
\usepackage{amssymb}
\usepackage{listings}
\usepackage{fancyhdr}
\usepackage{hyperref}
\usepackage{booktabs}
\usepackage{graphicx}
\usepackage{xfrac}
\usepackage[dot, autosize, outputdir="dotgraphs/"]{dot2texi}
\usepackage{tikz}
\usetikzlibrary{shapes}

\pagestyle{fancy}
\fancyhead{}
\fancyfoot{}
\rhead{\today}
\rfoot{\thepage}
\setlength\parskip{1em}
\setlength\parindent{1em}

%% Titel og forfatter
\title{Vurdering af projektstyringen i Tinglysningsprojektet}
\author{Maria Caroline Miller, 040779, twq135}

%% Start dokumentet
\begin{document}

%% Vis titel
\maketitle
\newpage

%% Vis indholdsfortegnelse
\tableofcontents
\newpage

\section{Indledning}
%Lidt om tinglysningsprojektet
Tinglysningsprojektet er et stort statsligt projekt, som er blevet igangsat for at gøre tinglysningen nemmere i fremtiden. Tinglysning er den offentlige registrering, efterprøvelse og offentliggørelse af rettigheder over fast ejendom, som huse, biler, løsøre, ægtepagter og andelsbeviser. Tinglysningsprojektet handler om at samle disse oplysninger i et elektronisk system, istedet for den gamle model, hvor det blev skrevet ned i bøger. 

Samtidig med at Domsstolsstyrelsen står for at gennemføre Tinglysningsprojektet, er man også igang domstolsreformen, der sørgede for oprettelsen af Tinglysningsretten i Hobro. Hermed vil tinglysningen blive flyttet fra de enkelte byretter, og fremefter ligge samlet i Hobro. 


\subsection{Virksomhedsbeskrivelse}
%Domstolsstyrelsen
%Tinglysning på den gamle facon
Domstolsstyrelsen, som er den direkte ejer af projektet, er en styrelse under Justitsministeriet. Det er Domstolsstyrelsen, der administrerer og udvikler de danske domstole. Styrelsen ledes af en bestyrelse og en daglig ledelse. Bestyrelsen består af en højesteretsdommer, to landsdommere, to byretsdommere, en repræsentation for det øvrige juridiske personale ved domstolene, to repræsentanter for det administrative personale ved domstolene, en advokat, og to medlemmer med særlig ledelsesmæssig og samfundsmæssig indsigt. Den daglige ledelse varetages af en direktør og en stab.

\subsection{Tinglysning før projektet}
Inden det nye it-system kom frem blev tinglysning foretaget i de forskellige byretter. Man havde i hver retskreds en bog med tinglysninger, som blev holdt ajour i hånden. Dette var selvfølgelig tidskrævende at alting skulle opdateres i hånden. Det var også besværligt når ting skulle slåes op, da man skulle finde den rette bog osv. Derfor besluttede man at lave dette projekt, for at få centraliseret tingene et sted, og gøre det nemmere at arbejde med.

\section{Problemformulering}
%Evt. er det endelige produkt blevet afleveret
%Taget fra opgaven
Problemformuleringen til denne opgave er:
\begin{itemize}
\item Diskut\ér i hvilket omfang Domstolsstyrelsen har opnået succes med projektet ud fra en vurdering af i hvilken grad forventede costs og benefits er realiseret.
\item Analys\ér hvilke forhold der har været afgørende for i hvilken grad costs og benefits er blevet realiseret hos Domstolsstyrelsen.
\item Vurd\ér hvilket risikoniveau projektet har kørt med, og vurder hvilke tiltag Domstolsstyrelsen kunne have gennemført med henblik på at mindske risikoniveauet.

\subsection{Afgrænsning}
%Faser fra Marchewka? Christian har brugt fase 1, 2 og 5.
%Måske bruge modenhedsanalysen?

%Følgende teoretiske materiale bruges: s. 50-53 + s. 212

\section{Teori og metode}
%Risikostyring, fordele (TBO - Total Benefit of Ownership), udgifter (TCO - Total Cost of Ownership) og værditilførelse (MOV - Measurable Organisation Value)

%Brug teori-billedet han altid viser i slides

%PLC og SDLC, s. 36, fig. 2.1

%Business case: figur 2.3, s. 43

\subsection{Risikohåndtering}
%Risici bør have en så kontrolleret indvirkning som muligt. Negative risici bør kontrolleres, så de har en minimal påvirkning på resten af projektet.

%Marchewka (205-233):
%Risikocyklus

%Risk Management - Marchewka (205-239)

%s. 212, figur 8.2: Identificering af risici og opdeling


\subsection{MOV, TBO- og TCO-modellerne}
%TBO og TCO: Marchewka s.52-53.
%MOV: Marchewka s. 44-50. fig 2.2 s. 50
%Marchewka s. 37

\section{Diskussion}
%Er succeskriterierne nået?
%MOV
%Målbare - slides 2, s. 8

%Opgaven om successkriterier

%Har offentlige projekter TCO og TBO?


\section{Analyse og vurdering}
%Hvilke forhold har været afgørende for realisering?

%Fase 5 side 39


\subsection{Udgifter og fordele}
%TCO: Direct costs, ongoing costs, indirect costs

\subsection{Risikovurdering}
%Hvilket risikoniveau har projektet kørt med?
%Mediestrategi?

%Risk management processes s. 209, figur 8.1

%Risikoniveau: tabel med elementer fra 5 lags framework
%Typen: figur 8.4, s.212


\subsubsection{Risikoidentifikation}

\subsubsection{Risikovurdering}

%Risikomatrice christians opgave s. 16

\subsubsection{Risikostrategi}

%s. 210

\subsubsection{Risikoevaluering}

%s. 210



\section{Konklusion}


\section{Litteraturliste}
\begin{thebibliography}{9}

\bibitem{Krav}
  Devoteam Consulting
  \emph{Bilag 2: Kravspecifikation for det kommende
  system til elektronisk tinglysning}
  26. juni 2006

\bibitem{Marchewka}
  Jack T. Marchewka,
  \emph{\LaTeX: Information Technology Project Management}.
  John Wiley \& Sons, Inc.
  3nd Edition,
  2010.

\bibitem{Rigs}
  Rigsrevisionen,
  \emph{Beretning til Statsrevisorerne om det digitale   
  tinglysningsprojekt}
  August, 2010

\bibitem{Tid}
  \emph{Bilag 1: Hovedtidsplan}
  26. juni 2006

\bibitem{Wiki}
  \emph{http://da.wikipedia.org/wiki/Domstolsstyrelsen}
  3. juni 2014, kl. 22.50

\end{thebibliography}

\section{Bilag}
\subsection{Bilag A - Interessent analyse}






\end{document}
