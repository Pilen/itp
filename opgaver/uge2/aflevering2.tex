\documentclass[10pt,a4paper,danish]{article}
\usepackage[danish]{babel}
\usepackage[utf8]{inputenc}
\usepackage{amsmath}
\usepackage{amssymb}
\usepackage{listings}
\usepackage{fancyhdr}
\usepackage[hidelinks]{hyperref}
\usepackage{booktabs}
\usepackage{graphicx}
\usepackage{xfrac}
\usepackage[dot, autosize, outputdir="dotgraphs/"]{dot2texi}
\usepackage{tikz}
\usepackage{ulem}
\usetikzlibrary{shapes}

\pagestyle{fancy}
\fancyhead{}
\fancyfoot{}
\rhead{\today}
\rfoot{\thepage}
\setlength\parskip{1em}
\setlength\parindent{1em}

%% Titel og forfatter
\title{Ugeopgave 2: Interessentanalyse}
\author{Søren Pilgård, 190689, vpb984\\
Caroline Miller, 04071979, twq135\\
René Løwe Jacobsen, 070192, vlx198\\
Paw Saabye Petersen, 110788, vwn245}

%% Start dokumentet
\begin{document}

%% Vis titel
\maketitle
\newpage

%% Vis indholdsfortegnelse
%%\tableofcontents
%%\newpage

\section{Identificering af interessenter}
% Domstolsstyrelsen er under
% Justitsministeriet
% Tinglysningsretten
% tinglysningsretten i hobro
% tinglysningsudvalget - Domstolsstyrelsen, finansministeriet, finansrådet, reelkredit rådet, advokatrådet, dansk ejendomsmæglerforening
% finansudvalget

% Statsrevisoren
% De ansatte i kommunerne der stod for tinglysningen i den enkelte retskreds.

% BRUGERENE:
% Banker og andre kreditorer (den finansielle sektor)
% ejendomsmæglere
% advokater
% borgere

% Deloitte
% Leverandøren
% ekstern leverandørstyringskonsulent
% projektledelsen
% csc


% Reelle:
Vi har identificeret følgende 8 interessenter:
\begin{itemize}
\item Borgerne
\item Finansielle sektor
\item Ejendomsmæglere og advokater
\item Leverandøren - CSC
\item Domsstolsstyrelsen - projektejer
\item Tinglysningsudvalget
\item Tinglysningsretten i Hobro
\item Projektledelsen
\end{itemize}

\section{Beskrivelse}
\subsection{Borgerne}
% Søren
Borgerne har en indirekte interesse i tinglysnings projektet, selv om det er
deres informationer, eller informationer om dem, der lagres interagere de fleste ikke
selv direkte med den. Borgerens interesse består dermed primært i at
oplysningerne er tilgængelige for de rette parter så de ikke risikere at komme i
klemme eller miste penge som det skete for flere personer der mistede penge på
rentetab. Borgerne kan derudover have en interesse i at det kun er de rette
parter der har adgang til oplysningerne så alle og enhver ikke kan gå ind og
aflæse privat information om dem.
Derudover er det borgernes penge der bruges på projektet indirekte via skatten.
Dermed er det i borgernes interesse at projektet holdes på rette spor og inden
for budgettet.
Så længe disse interesser bliver overholdt vil den almindelige borger næppe
have den store interesse eller aktivitet omkring projektet.
Og det er også disse emner der vil kunne opstå en eventuel konflikt omkring selv
om borgerne reelt ikke har den store magt eller indflydelse på projektet.

Det er naturligvis umuligt at indrage alle borgere og få alles input.
For at give alle lov til at komme med input kan man afholde offentlige
høringer. Derudover kan man udvælge en række representanter som

\subsection{Finansielle sektor}
% Søren

Den finansielle sektor er en anden gruppe af brugere af tinglysningsprojektet.
Gruppen består af de banker og kreditorer der står for udlånet til borgeren.

Gruppen har interesse i nemt og billigt at kunne tilgå og tilføje posteringer i systemet.
Det er vigtigt for bankerne at oplysningerne om lån er korrekte og tilgængelige.
Bankerne vil stå for en stor del af aktiviteterne omkring brugen af systemet og
ser formentlig digitaliserings fordele i form af hurtigere proces tid.

Den finansielle sektor har ikke en direkte magt i forhold til projektet, men er alligevel
en vigtig part da det netop er deres penge og udlån der registreres. De udgør
dermed en politisk part over for projektet og beslutningstagerne.

De primære konflikter vil formentlig opstå såfremt systemet ikke er
tilgængeligt eller hvis banken skal betale et beløb de ikke er villige til for at benytte systemet.

Den finansielle sektor bør inddrages via dialog. Eventuelt kan man finde
repræsentanter via en branche forening hvorigennem feedback kan hentes.

\subsection{Ejendomsmæglere og advokater}
% Søren


\subsection{Leverandøren - CSC}
% René


\subsection{Domsstolsstyrelsen - projektejer}
% Caro
Da Domstolsstyrelsen er ejeren af projektet har de store interesser i projektet. Det er dem der skal sørge for at udvikle foranalyse og forberede hele projektet. Derfor har de selvsagt også en hel del magt over selve projektet, da det jo er reelt er dem der udvikler og gennemfører hele projektet.

De har et højt aktivitetsniveau, og er med i de fleste aspekter af projekter, og sørger for at deres analyse og forskrifter bliver overholdt. Domstolsstyrelsen kan komme i konflikter med andre interessenter, hvis de ikke føler at deres forskrifter bliver overholdt, eller de ikke føler at andre lever op til de krav de sætter til dem. Domstolsstyrelsen sørger selv for at inddrage sig selv i hele processen ved at være den ledende del af projektet, og de skal sørge for at håndtere hele arbejdet.


\subsection{Tinglysningsudvalget}
% Caro
% tinglysningsudvalget - Domstolsstyrelsen, finansministeriet, finansrådet, reelkredit rådet, advokatrådet, dansk ejendomsmæglerforening

Tinglysningsudvalget er et udvalgt nedsat af vigtige repræsentanter for alle interessenter. Der sidder blandt andet repræsentanter for Domsstolsstyrelsen, Finansministeriet, Finansrådet, Realkreditrådet, Advokatrådet og Dansk Ejendomsmæglerforening. Udvalget blev nedsat for at komme med et forslag til en modernisering og effektivisering af tinglysningsopgaven

De er derfor en vigtig interessent i projektet, da det er dem der har udviklet den indledende ide til projektet. Deres mål er at få udviklet projektet, og at få og de er interesseret i at se deres indledende undersøgelse gennemført.

De vil stadig være aktive i projektet, da udvalget netop består af en blanding af de fleste interessenter i projektet, og de derfor gerne vil være en del i de forskellige aspekter, og sørge for at hver enkelts specifikke krav kommer igennem.

Interessenten har magt, da de sammen med Domsstolsstyrelsen står for at gennemføre projektet. De er derfor en vigtig del af hele projektet.

Konfliktmæssigt er der store chancer for konflikter internt i udvalget, da der sidder repræsentanter for alle dele af projektet, og der er store chancer for at den enes krav måske vanskeliggør andre krav. De har også høje meninger omkring hvordan projektet skal udføres, og hvis de ikke mener at arbejdet bliver udført ordentligt og rigtigt, kan der kommer konflikter.

Interessenten er inddraget i projektet gennem møder i projektperioden. Man må formode at det meste arbejde lægges over til Domsstolsstyrelsen, men at beslutninger tages ved møder i dette udvalg.

\subsection{Tinglysningsretten i Hobro}
% René


\subsection{Projektledelsen}
% Caro
Man må formode at Domstolsstyrelsen har hyret en projektleder til at styre projektet for dem. Denne person har nok en del nøglepersoner, som står for at håndtere forskellige dele af projektet og vende tilbage til projektlederen. Dette er projektledelsen.

Deres mål og interesser er at gennemføre projektet så godt som muligt, da det er deres job, og de gerne vil have et succesfyldt projekt på CV'et. De har nok ikke "`personlige"' interesser i selve projektet, hvis de er ansat udefra.

Deres aktivitetsniveau er højt, fordi det er dem der reelt står for gennemførelse af projektet. Det er et fuldtidsjob for dem.

De har en smule magt på det daglige arbejde, men har ikke nogen reel magt over de store udsving i projektet, da arbejder ud fra en projektbeskrivelse.

Der kan være konflikter i forbindelse med forsinkelser af arbejdet. Dette kan skabe konflikter til Domsstolsstyrelsen, som jo ønsker at få projektet færdig til tiden. Derudover kan der komme konflikter i forbindelse med arbejdets udførelse, arbejdet ikke bliver udført efter forskrifterne.

Projektledelsen håndteres som enhver anden ansat fordi de får løn, og har en kontrakt. De får højest sandsynligt en rigtig god løn, og gode fordele, sådan at man ikke risikerer at den pludselig siger op midt i arbejdet. Samtidig er det muligt at projektlederen også er med til de indledende møder, for at inddrage personen mere i det forestående arbejde.

\end{document}
