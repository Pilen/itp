\documentclass[10pt,a4paper,danish]{article}
\usepackage[danish]{babel}
\usepackage[utf8]{inputenc}
\usepackage{amsmath}
\usepackage{amssymb}
\usepackage{listings}
\usepackage{fancyhdr}
\usepackage[hidelinks]{hyperref}
\usepackage{booktabs}
\usepackage{graphicx}
\usepackage{xfrac}
\usepackage[dot, autosize, outputdir="dotgraphs/"]{dot2texi}
\usepackage{tikz}
\usepackage{ulem}
\usetikzlibrary{shapes}

\pagestyle{fancy}
\fancyhead{}
\fancyfoot{}
\rhead{\today}
\rfoot{\thepage}
\setlength\parskip{1em}
\setlength\parindent{1em}

%% Titel og forfatter
\title{Ugeopgave 2: Interessentanalyse}
\author{Søren Pilgård, 190689, vpb984\\
Caroline Miller, 04071979, twq135\\
René Løwe Jacobsen, 070192, vlx198\\
Paw Saabye Petersen, 110788, vwn245}

%% Start dokumentet
\begin{document}

%% Vis titel
\maketitle
\newpage

%% Vis indholdsfortegnelse
%%\tableofcontents
%%\newpage

\section{Identificering af interessenter}
% Domstolsstyrelsen er under
% Justitsministeriet
% Tinglysningsretten
% tinglysningsretten i hobro
% tinglysningsudvalget - Domstolsstyrelsen, finansministeriet, finansrådet, reelkredit rådet, advokatrådet, dansk ejendomsmæglerforening
% finansudvalget

% Statsrevisoren
% De ansatte i kommunerne der stod for tinglysningen i den enkelte retskreds.

% BRUGERENE:
% Banker og andre kreditorer (den finansielle sektor)
% ejendomsmæglere
% advokater
% borgere

% Deloitte
% Leverandøren
% ekstern leverandørstyringskonsulent
% projektledelsen
% csc


% Reelle:
Vi har identificeret følgende 8 interessenter:
\begin{itemize}
\item Borgerne
\item Finansielle sektor
\item Ejendomsmæglere og advokater
\item Leverandøren - CSC
\item Domsstolsstyrelsen - projektejer
\item Tinglysningsudvalget
\item Tinglysningsretten i Hobro
\item Projektledelsen
\end{itemize}

\section{Beskrivelse}
\subsection{Borgerne}
% Søren
Borgerne har en indirekte interesse i tinglysnings projektet, selv om det er
deres informationer, eller informationer om dem, der lagres interagere de fleste ikke
selv direkte med den. Borgerens interesse består dermed primært i at
oplysningerne er tilgængelige for de rette parter så de ikke risikere at komme i
klemme eller miste penge som det skete for flere personer der mistede penge på
rentetab. Borgerne kan derudover have en interesse i at det kun er de rette
parter der har adgang til oplysningerne så alle og enhver ikke kan gå ind og
aflæse privat information om dem.
Derudover er det borgernes penge der bruges på projektet indirekte via skatten.
Dermed er det i borgernes interesse at projektet holdes på rette spor og inden
for budgettet.
Så længe disse interesser bliver overholdt vil den almindelige borger næppe
have den store interesse eller aktivitet omkring projektet.
Og det er også disse emner der vil kunne opstå en eventuel konflikt omkring selv
om borgerne reelt ikke har den store magt eller indflydelse på projektet.

Det er naturligvis umuligt at indrage alle borgere og få alles input.
For at give alle lov til at komme med input kan man afholde offentlige
høringer. Derudover kan man udvælge en række representanter som

\subsection{Finansielle sektor}
% Søren


\subsection{Ejendomsmæglere og advokater}
% Søren


\subsection{Leverandøren - CSC}
% René


\subsection{Domsstolsstyrelsen - projektejer}
% Caro
Da Domstolsstyrelsen er ejeren af projektet har denne selvsagt meget magt over projektet. Det er dem, der har det endelige ansvar for projektet.


\subsection{Tinglysningsudvalget}
% Caro


\subsection{Tinglysningsretten i Hobro}
% René


\subsection{Projektledelsen}
% Caro


\end{document}
