\documentclass[10pt,a4paper,danish]{article}
\usepackage[danish]{babel}
\usepackage[utf8]{inputenc}
\usepackage{amsmath}
\usepackage{amssymb}
\usepackage{listings}
\usepackage{fancyhdr}
\usepackage[hidelinks]{hyperref}
\usepackage{booktabs}
\usepackage{graphicx}
\usepackage{xfrac}
\usepackage[dot, autosize, outputdir="dotgraphs/"]{dot2texi}
\usepackage{tikz}
\usepackage{ulem}
\usetikzlibrary{shapes}

\pagestyle{fancy}
\fancyhead{}
\fancyfoot{}
\rhead{\today}
\rfoot{\thepage}
\setlength\parskip{1em}
\setlength\parindent{1em}

%% Titel og forfatter
\title{Ugeopgave 2: Interessentanalyse}
\author{Søren Pilgård, 190689, vpb984\\
Caroline Miller, 04071979, twq135\\
René Løwe Jacobsen, 070192, vlx198\\
Paw Saabye Petersen, 110788, vwn245}

%% Start dokumentet
\begin{document}

%% Vis titel
\maketitle
\newpage

%% Vis indholdsfortegnelse
%%\tableofcontents
%%\newpage

\section{Identificering af interessenter}
Domstolsstyrelsen er under
Justitsministeriet
Tinglysningsretten
tinglysningsretten i hobro
tinglysningsudvalget - Domstolsstyrelsen, finansministeriet, finansrådet, reelkredit rådet, advokatrådet, dansk ejendomsmæglerforening
finansudvalget

Statsrevisoren
De ansatte i kommunerne der stod for tinglysningen i den enkelte retskreds.

BRUGERENE:
Banker og andre kreditorer (den finansielle sektor)
ejendomsmæglere
advokater
borgere

Deloitte
Leverandøren
ekstern leverandørstyringskonsulent
projektledelsen
csc


Reelle:
1. Borgerne
2. Finansielle sektor
3. Ejendomsmæglere og advokater
4. Leverandøren - CSC
5. Domsstolsstyrelsen - projektejer
6. Tinglysningsudvalget
7. Tinglysningsretten i Hobro
8. Projektledelsen


\section{Beskrivelse}

\end{document}
