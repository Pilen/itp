\documentclass[10pt,a4paper,danish]{article}
\usepackage[danish]{babel}
\usepackage[utf8]{inputenc}
\usepackage{amsmath}
\usepackage{amssymb}
\usepackage{listings}
\usepackage{fancyhdr}
\usepackage[hidelinks]{hyperref}
\usepackage{booktabs}
\usepackage{graphicx}
\usepackage{xfrac}
\usepackage[dot, autosize, outputdir="dotgraphs/"]{dot2texi}
\usepackage{tikz}
\usepackage{ulem}
\usetikzlibrary{shapes}

\pagestyle{fancy}
\fancyhead{}
\fancyfoot{}
\rhead{\today}
\rfoot{\thepage}
\setlength\parskip{1em}
\setlength\parindent{1em}

%% Titel og forfatter
\title{Ugeopgave 2: Interessentanalyse}
\author{Søren Pilgård, 190689, vpb984\\
Caroline Miller, 04071979, twq135\\
René Løwe Jacobsen, 070192, vlx198\\
Paw Saabye Petersen, 110788, vwn245}

%% Start dokumentet
\begin{document}

%% Vis titel
\maketitle
\newpage

%% Vis indholdsfortegnelse
%%\tableofcontents
%%\newpage

\section{Identificering af interessenter}
% Domstolsstyrelsen er under
% Justitsministeriet
% Tinglysningsretten
% tinglysningsretten i hobro
% tinglysningsudvalget - Domstolsstyrelsen, finansministeriet, finansrådet, reelkredit rådet, advokatrådet, dansk ejendomsmæglerforening
% finansudvalget

% Statsrevisoren
% De ansatte i kommunerne der stod for tinglysningen i den enkelte retskreds.

% BRUGERENE:
% Banker og andre kreditorer (den finansielle sektor)
% ejendomsmæglere
% advokater
% borgere

% Deloitte
% Leverandøren
% ekstern leverandørstyringskonsulent
% projektledelsen
% csc


% Reelle:
Vi har identificeret følgende 8 interessenter:
\begin{itemize}
\item Borgerne
\item Finansielle sektor
\item Ejendomsmæglere og advokater
\item Leverandøren - CSC
\item Domsstolsstyrelsen - projektejer
\item Tinglysningsudvalget
\item Tinglysningsretten i Hobro
\item Projektledelsen
\end{itemize}

\section{Beskrivelse}
\subsection{Borgerne}
% Søren
Borgerne har en indirekte interesse i tinglysnings projektet, selv om det er
deres informationer, eller informationer om dem, der lagres interagere de fleste ikke
selv direkte med den. Borgerens interesse består dermed primært i at
oplysningerne er tilgængelige for de rette parter så de ikke risikere at komme i
klemme eller miste penge som det skete for flere personer der mistede penge på
rentetab. Borgerne kan derudover have en interesse i at det kun er de rette
parter der har adgang til oplysningerne så alle og enhver ikke kan gå ind og
aflæse privat information om dem.
Derudover er det borgernes penge der bruges på projektet indirekte via skatten.
Dermed er det i borgernes interesse at projektet holdes på rette spor og inden
for budgettet.
Så længe disse interesser bliver overholdt vil den almindelige borger næppe
have den store interesse eller aktivitet omkring projektet.
Og det er også disse emner der vil kunne opstå en eventuel konflikt omkring selv
om borgerne reelt ikke har den store magt eller indflydelse på projektet.

Det er naturligvis umuligt at indrage alle borgere og få alles input.
For at give alle lov til at komme med input kan man afholde offentlige
høringer. Derudover kan man udvælge en række representanter som

\subsection{Finansielle sektor}
% Søren


\subsection{Ejendomsmæglere og advokater}
% Søren


\subsection{Leverandøren - CSC}
% René


\subsection{Domsstolsstyrelsen - projektejer}
% Caro
Da Domstolsstyrelsen er ejeren af projektet har de store interesser i projektet. Det er dem der skal sørge for at udvikle foranalyse og forberede hele projektet. Derfor har de selvsagt også en hel del magt over selve projektet, da det jo er reelt er dem der udvikler og gennemfører hele projektet.

De har et højt aktivitetsniveau, og er med i de fleste aspekter af projekter, og sørger for at deres analyse og forskrifter bliver overholdt. Domstolsstyrelsen kan komme i konflikter med andre interessenter, hvis de ikke føler at deres forskrifter bliver overholdt, eller de ikke føler at andre lever op til de krav de sætter til dem. Domstolsstyrelsen sørger selv for at inddrage sig selv i hele processen ved at være den ledende del af projektet, og de skal sørge for at håndtere hele arbejdet.


\subsection{Tinglysningsudvalget}
% Caro
% tinglysningsudvalget - Domstolsstyrelsen, finansministeriet, finansrådet, reelkredit rådet, advokatrådet, dansk ejendomsmæglerforening

Tinglysningsudvalget er et udvalgt nedsat af vigtige repræsentanter for alle interessenter. Der sidder blandt andet repræsentanter for Domsstolsstyrelsen, Finansministeriet, Finansrådet, Realkreditrådet, Advokatrådet og Dansk Ejendomsmæglerforening. Udvalget blev nedsat for at komme med et forslag til en modernisering og effektivisering af tinglysningsopgaven

De er derfor en vigtig interessent i projektet, da det er dem der har udviklet den indledende ide til projektet. Deres mål er at få udviklet projektet, og at få og de er interesseret i at se deres indledende undersøgelse gennemført.

De vil stadig være aktive i projektet, da udvalget netop består af en blanding af de fleste interessenter i projektet, og de derfor gerne vil være en del i de forskellige aspekter, og sørge for at hver enkelts specifikke krav kommer igennem.

Interessenten har magt, da de sammen med Domsstolsstyrelsen står for at gennemføre projektet. De er derfor en vigtig del af hele projektet. 

Konfliktmæssigt er der store chancer for konflikter internt i udvalget, da der sidder repræsentanter for alle dele af projektet, og der er store chancer for at den enes krav måske vanskeliggør andre krav. De har også høje meninger omkring hvordan projektet skal udføres, og hvis de ikke mener at arbejdet bliver udført ordentligt og rigtigt, kan der kommer konflikter.

Interessenten er inddraget i projektet gennem møder i projektperioden. Man må formode at det meste arbejde lægges over til Domsstolsstyrelsen, men at beslutninger tages ved møder i dette udvalg.

\subsection{Tinglysningsretten i Hobro}
% René


\subsection{Projektledelsen}
% Caro


\end{document}
