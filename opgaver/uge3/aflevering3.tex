\documentclass[10pt,a4paper,danish]{article}
\usepackage[danish]{babel}
\usepackage[utf8]{inputenc}
\usepackage{amsmath}
\usepackage{amssymb}
\usepackage{listings}
\usepackage{fancyhdr}
\usepackage[hidelinks]{hyperref}
\usepackage{booktabs}
\usepackage{graphicx}
\usepackage{xfrac}
\usepackage[dot, autosize, outputdir="dotgraphs/"]{dot2texi}
\usepackage{tikz}
\usepackage{ulem}
\usetikzlibrary{shapes}

\pagestyle{fancy}
\fancyhead{}
\fancyfoot{}
\rhead{\today}
\rfoot{\thepage}
\setlength\parskip{1em}
\setlength\parindent{1em}

%% Titel og forfatter
\title{Ugeopgave 3: Risikoanalyse}
\author{Søren Pilgård, 190689, vpb984\\
Caroline Miller, 04071979, twq135\\
René Løwe Jacobsen, 070192, vlx198\\
Paw Saabye Petersen, 110788, vwn245}

%% Start dokumentet
\begin{document}

%% Vis titel
\maketitle
\newpage

%% Vis indholdsfortegnelse
%%\tableofcontents
%%\newpage

\section{Introduktion}
De vigtigste risici i Tinglysningsprojektet skal identificeres, og hver enkelt risiko skal have beskrevet konsekvenser og sandsynlighed. Til at beregne sandsynlighed og konsekvens har vi brugt skalaerne:0-100\% for sandsynlighed og 0-10 i konsekvens. Til sidst i opgaven kan ses en oversigt med en beregnet score (skala og skema er lånt fra Bishop, tabel 8.3, side 220 (i version 3).

\section{Identificering af risici}
Vi har identificeret følgende ? risici:
\begin{itemize}
\item Idriftsættelsesstrategien giver bagslag
\item Effektiviteten ved systemets ibrugtagen er ikke som forventet
\item Ansatte flytter ikke med til Hobro 
\item Nøglepersoner forlader projektet
\item Leverandør udfører ikke arbejdet hurtigt nok
\item Kunde kommer med for mange ændringer
\end{itemize}

\section{Beskrivelse af hver risiko}
% \subsection{Prisen på projektet holder ikke}
% \subsubsection{Årsag}
% Tidsplanen er ikke veldefineret og kan derfor indeholde overraskelser, der gør, at projektet bliver udskudt. Jo længere tid det tager at gennemfører projektet, jo dyrere er det.
% \subsubsection{Konsekvens}
% Projektet vil komme til at tage flere af skatteydernes penge·
% \subsubsection{Sandsynlighed}
% Ret sandsynligt %% Vi skal have en skala til sandsynlighed
% \subsubsection{Konsekvens}

% \subsubsection{Forebyggelse}




%\subsection{Tiden på projektet holder ikke}


\subsection{Idriftsættelsesstrategien giver bagslag}
\subsubsection{Årsag}
Til projektet har man valgt at bruge en big-bang implementeringsstrategi, hvor systemet skal tage 100\% over fra dag 1.
Med sådan et stort system er det sandsynligt, at det ikke vil fungere 100\% optimalt til at starte med eller at medarbejderne ikke er forberedt godt nok på systemet.
\subsubsection{Konsekvens}
Sagsbehandlingerne vil til at starte med bruge meget længere tid på enkelte sager end de ville med det gamle system. Det vil dog tage af efter et stykke tid.
\subsubsection{Sandsynlighed}
Vi har valgt at give det en sandsynlighed på 50\%.
\subsubsection{Konsekvens størrelse}
På en skala fra 1 til 10, har vi valgt at give denne 10.
\subsubsection{Risikoforebyggelse}
Ved at lave flere brugertests af systemet kan flere fejl blive fanget i processen og systemet kan blive forbedret, så de ansatte er mere konfortable med det. Det vil også gøre det nemmere at oplære de ansatte i, hvordan systemet virker. Pris ???.
\subsubsection{Konsekvensforebyggelse}
Ved at køre en pilottest på en mindre afdeling er det muligt at isolere fejl i systemet og ved oplæring af medarbejder. 
Dette gør, at andre medarbejdere hurtigere vil kunne ramme et godt effektivitetsniveau, når de tager systemet i brug.
Der vil stadig være en sandsynlighed for, at systemet ikke vil være effektivt, men det vil kun være for en lille del af de ansatte, der vil opleve problemet. Det vil formentlig komme til at koste nogle millioner, da det vil kræve, at to systemer kører samtidigt.

\subsection{Effektiviteten ved systemets ibrugtagen er ikke optimal}
%Rene


\subsection{Ansatte flytter ikke med til Hobro}
%Caro
\subsubsection{Årsag}
Arbejdet skal fremover samles i Hobro. Det er ikke sikkert at de ansatte som er uddannet i det gamle system, og bor rundt omkring i hele landet, gider flytte til Hobro.

\subsubsection{Konsekvens}
Der skal uddannes nyt personale i selve tinglysningsopgaven, samtidig med at man skal uddanne det gamle personale i det nye system.

\subsubsection{Sandsynlighed}
Sandsynligheden er ret stor for at ikke alle vælger at flytte med: 90\%.

\subsubsection{Konsekvens}
Ibrugtagningsperioden vil blive forlænget idet man nu udover at skulle indkøre det nye system, også skal indkøre nye medarbejdere i selve tinglysningen. Konsekvens score er 5.

\subsubsection{Forebyggelse}
Risikoen kan forebygges ved at give det eksisterende personale gode grunde til at flytte til Hobro. Højere løn, hjælp med at finde bolig etc. Risikoen kan også forebygges ved tidligt at undersøge hvormange der gerne vil flytte, og dermed tidligt kunne nå at uddanne nye medarbejdere i selve tinglysningen, så de kan uddannes i det nye system samtidig med de "`gamle"' medarbejdere. Sandsynligheden for at ikke alle flytter med vil nok stadig være den samme, men konsekvens scoren vil falde til 3.

\subsection{Nøglepersoner forlader projektet}
%Caro
\subsubsection{Årsag}
En vigtig del af projektgruppen kan forlade projektet grundet et bedre jobtilbud, eller lysten til at lave noget andet.

\subsubsection{Konsekvens}
Hvis det er en af nøglepersonerne kan den del, som personen stod for, nemt komme til at hænge, da det kan være svært at overtage. 

\subsubsection{Sandsynlighed}
Sandsynligheden for dette er omkring 40\%. Da Tinglysningsprojektet er et meget stort og langt projekt, så kan det være svært ikke at få lyst til nye udfordringer undervejs.

\subsubsection{Konsekvens}
Konsekvensen kan svinge rigtig meget, da det afhænger af hvilke kontraktlige forpligtelser man har lavet til at starte med. Er der f.eks. to måneders opsigelse (i modsætning til normalt en), så personen bedre kan nå at oplære en ny? Kan man ved at tilbyde mere i løn eller lignende få personen til at blive? Hvis personen forlader projektet er konsekvensen nok 4.

\subsubsection{Forebyggelse}




\subsection{Leverandør udfører ikke arbejdet hurtigt nok}
%Rene
\subsubsection{Årsag}
Tidsplanen er ikke veldefineret og kan derfor indeholde overraskelser, der gør, at projektet bliver udskudt. Jo længere tid det tager at gennemfører projektet, jo dyrere er det.
\subsubsection{Konsekvens}
Projektet vil komme til at tage flere af skatteydernes penge.
\subsubsection{Sandsynlighed}

\subsubsection{Konsekvens størrelse}

\subsubsection{Risikoforebyggelse}


\subsection{Kunde kommer med for mange ændringer}
%Caro
\subsubsection{Årsag}


\subsubsection{Konsekvens}


\subsubsection{Sandsynlighed}


\subsubsection{Konsekvens}


\subsubsection{Forebyggelse}


\begin{itemize}
\item Prisen på projektet holder ikke
\item Tiden på projektet holder ikke
\item Idriftsættelsesstrategien giver bagslag
\item Effektiviteten ved systemets ibrugtagen er ikke som forventet
\item Ansatte flytter ikke med til Hobro 
\item Nøglepersoner forlader projektet
\item Leverandør udfører ikke arbejdet hurtigt nok
\item Kunde kommer med for mange ændringer
\end{itemize}


\section{Risiko analyse i skema}
\begin{tabular}{l|l|l|l}
           & 0-100\%       & 0-10       & P*I   \\
Risiko     & Sandsynlighed & Konsekvens & Score \\\hline
Pris       &
Tid        &
Idriftsættelse &
Effektivitet   &
Flytning   &
Nøglepersoner &
Leverandør &
Ændringsforslag &
\end{tabular}


\end{document}
