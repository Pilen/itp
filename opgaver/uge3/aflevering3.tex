\documentclass[10pt,a4paper,danish]{article}
\usepackage[danish]{babel}
\usepackage[utf8]{inputenc}
\usepackage{amsmath}
\usepackage{amssymb}
\usepackage{listings}
\usepackage{fancyhdr}
\usepackage[hidelinks]{hyperref}
\usepackage{booktabs}
\usepackage{graphicx}
\usepackage{xfrac}
\usepackage[dot, autosize, outputdir="dotgraphs/"]{dot2texi}
\usepackage{tikz}
\usepackage{ulem}
\usetikzlibrary{shapes}

\pagestyle{fancy}
\fancyhead{}
\fancyfoot{}
\rhead{\today}
\rfoot{\thepage}
\setlength\parskip{1em}
\setlength\parindent{1em}

%% Titel og forfatter
\title{Ugeopgave 3: Risikoanalyse}
\author{Søren Pilgård, 190689, vpb984\\
Caroline Miller, 04071979, twq135\\
René Løwe Jacobsen, 070192, vlx198\\
Paw Saabye Petersen, 110788, vwn245}

%% Start dokumentet
\begin{document}

%% Vis titel
\maketitle
\newpage

%% Vis indholdsfortegnelse
%%\tableofcontents
%%\newpage

\section{Identificering af risici}
Vi har identificeret følgende ? risici:
\begin{itemize}
\item Prisen på projektet holder ikke
\item Tiden på projektet holder ikke
\item Idriftsættelsesstrategien giver bagslag
\item Effektiviteten ved systemets ibrugtagen er ikke som forventet
\item Ansatte flytter ikke med til Hobro 
\item Nøglepersoner forlader projektet
\item Leverandør udfører ikke arbejdet hurtigt nok
\item Kunde kommer med for mange ændringer
\end{itemize}

\section{Beskrivelse af hver risiko}
\subsection{Prisen på projektet holder ikke}
\subsubsection{Årsag}
Tidsplanen er ikke veldefineret og kan derfor indeholde overraskelser, der gør, at projektet bliver udskudt. Jo længere tid det tager at gennemfører projektet, jo dyrere er det.
\subsubsection{Konsekvens}
Projektet vil komme til at tage flere af skatteydernes penge.
\subsubsection{Sandsynlighed}

\subsubsection{Konsekvens størrelse}

\subsubsection{Risikoforebyggelse}




\subsection{Tiden på projektet holder ikke}





\subsection{Idriftsættelsesstrategien giver bagslag}
\subsubsection{Årsag}
Til projektet har man valgt at bruge en big-bang implementeringsstrategi, hvor systemet skal tage 100\% over fra dag 1.
Med sådan et stort system er det sandsynligt, at det ikke vil fungere 100\% optimalt til at starte med eller at medarbejderne ikke er forberedt godt nok på systemet.
\subsubsection{Konsekvens}
Sagsbehandlingerne vil til at starte med bruge meget længere tid på enkelte sager end de ville med det gamle system. Det vil dog tage af efter et stykke tid.
\subsubsection{Sandsynlighed}
Vi har valgt at give det en sandsynlighed på 50\%.
\subsubsection{Konsekvens størrelse}
På en skala fra 1 til 10, har vi valgt at give denne 10.
\subsubsection{Risikoforebyggelse}
Ved at lave flere brugertests af systemet kan flere fejl blive fanget i processen og systemet kan blive forbedret, så de ansatte er mere konfortable med det. Prisen her vil være ??????.
\subsubsection{Konsekvensforebyggelse}
Ved at køre en pilottest på en mindre afdeling er det muligt at isolere fejl i systemet og ved oplæring af medarbejder. 
Dette gør, at andre medarbejdere hurtigere vil kunne ramme et godt effektivitetsniveau, når de tager systemet i brug.
Der vil stadig være en sandsynlighed for, at systemet ikke vil være effektivt, men det vil kun være for en lille del af de ansatte, der vil opleve problemet. Det vil formentlig komme til at koste nogle millioner, da det vil kræve, at to systemer kører samtidigt.

\subsection{Effektiviteten ved systemets ibrugtagen er ikke optimal}
\subsubsection{Årsag}



\subsection{Ansatte flytter ikke med til Hobro}
\subsection{Nøglepersoner forlader projektet}
\subsection{Leverandør udfører ikke arbejdet hurtigt nok}
\subsection{Kunde kommer med for mange ændringer}

\end{document}
