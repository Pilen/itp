\documentclass[10pt,a4paper,danish]{article}
\usepackage[danish]{babel}
\usepackage[utf8]{inputenc}
\usepackage{amsmath}
\usepackage{amssymb}
\usepackage{listings}
\usepackage{fancyhdr}
\usepackage[hidelinks]{hyperref}
\usepackage{booktabs}
\usepackage{graphicx}
\usepackage{xfrac}
\usepackage[dot, autosize, outputdir="dotgraphs/"]{dot2texi}
\usepackage{tikz}
\usepackage{ulem}
\usetikzlibrary{shapes}

\pagestyle{fancy}
\fancyhead{}
\fancyfoot{}
\rhead{\today}
\rfoot{\thepage}
\setlength\parskip{1em}
\setlength\parindent{1em}

%% Titel og forfatter
\title{Ugeopgave 3: Risikoanalyse}
\author{Søren Pilgård, 190689, vpb984\\
Caroline Miller, 04071979, twq135\\
René Løwe Jacobsen, 070192, vlx198\\
Paw Saabye Petersen, 110788, vwn245}

%% Start dokumentet
\begin{document}

%% Vis titel
\maketitle
\newpage

%% Vis indholdsfortegnelse
%%\tableofcontents
%%\newpage

\section{Introduktion}
De vigtigste risici i Tinglysningsprojektet (Ting) skal identificeres, og hver enkelt risiko skal have beskrevet konsekvenser og sandsynlighed. Til at beregne sandsynlighed og konsekvens har vi brugt skalaerne: 0-100\% for sandsynlighed og 0-10 i konsekvens. Til sidst i opgaven kan ses en oversigt med en beregnet score (skala og skema er lånt fra Bishop, tabel 8.3, side 220 (i version 3)).

\section{Identificering af risici}
Vi har identificeret følgende 6 risici:
\begin{itemize}
\item Idriftsættelsesstrategien giver bagslag
\item Effektiviteten ved systemets ibrugtagen er ikke som forventet
\item Ansatte flytter ikke med til Hobro 
\item Nøglepersoner forlader projektet
\item Leverandør udfører ikke arbejdet hurtigt nok
\item Kunde kommer med for mange ændringer
\end{itemize}

\section{Beskrivelse af hver risiko}
% \subsection{Prisen på projektet holder ikke}
% \subsubsection{Årsag}
% Tidsplanen er ikke veldefineret og kan derfor indeholde overraskelser, der gør, at projektet bliver udskudt. Jo længere tid det tager at gennemfører projektet, jo dyrere er det.
% \subsubsection{Konsekvens}
% Projektet vil komme til at tage flere af skatteydernes penge·
% \subsubsection{Sandsynlighed}
% Ret sandsynligt %% Vi skal have en skala til sandsynlighed
% \subsubsection{Konsekvens}

% \subsubsection{Forebyggelse}




%\subsection{Tiden på projektet holder ikke}


\subsection{Idriftsættelsesstrategien giver bagslag}
\subsubsection{Årsag}
Til projektet har man valgt at bruge en big-bang implementeringsstrategi, hvor systemet skal tage 100\% over fra dag 1.
Med sådan et stort system er det sandsynligt, at det ikke vil fungere 100\% optimalt til at starte med eller at medarbejderne ikke er forberedt godt nok på systemet. 

\subsubsection{Konsekvens}
Sagsbehandlerne vil til at starte med bruge meget længere tid på enkelte sager end de ville med det gamle system. Der er en chance for at hvis systemet ikke fungerer ordentligt, vil der være problemer for tinglysningskunderne.

\subsubsection{Sandsynlighed}
Da dette er et ret stort og komplekst system er sandsynligheden 70\%. Havde projektet været mindre ville procentsatsen også være mindre.

\subsubsection{Konsekvens størrelse}
Da det gamle system vil blive udfaset helt og der satses 100\% på det nye system, så vil konsekvensniveauet være 9, da det vil være fatalt, hvis det nye system ikke virker ordentligt fra start.

\subsubsection{Risikoforebyggelse}
Ved at lave flere brugertests af systemet kan flere fejl blive fanget i processen og systemet kan blive forbedret, så de ansatte er mere komfortable med det. Det vil også gøre det nemmere at oplære de ansatte i, hvordan systemet virker. Omkostningerne vil være højere, da det indledende arbejde vil tage længere tid, og brugertests koster mange penge at udføre.

\subsubsection{Konsekvensforebyggelse}
Ved at køre en pilottest på en mindre afdeling er det muligt at isolere fejl i systemet og fejl i oplæring af medarbejdere og andre medarbejdere vil hurtigere kunne ramme et godt effektivitetsniveau, når de tager systemet i brug.
Der vil stadig være en sandsynlighed for, at systemet ikke vil være effektivt, men det er kun en lille del af de ansatte, der vil opleve problemet. Det vil formentlig komme til at koste nogle millioner, da det vil kræve, at to systemer kører samtidigt.

\subsection{Effektiviteten ved systemets ibrugtagen er ikke optimal}
%Rene
\subsubsection{Årsag}
Forventningerne til de ansatte kan være for høje. Det tager tid at sætte sig ind i IT-systemer og til at starte med kan effektiviteten godt falde inden den reelle gevinst kommer. Det kan også være, at undervisningen i det nye system ikke er god nok.

\subsubsection{Konsekvens}
De enkelte sager vil tage længere tid at behandle til at starte med, og der kan ske fejl.

\subsubsection{Sandsynlighed}
Da det er et helt nyt system er sandsynligheden ret stor: 90\%.

\subsubsection{Konsekvens størrelse}
Da det burde være kendt, at produktiviten kan falde i en periode ved implementeringen af et nyt system, så bør det være medregnet i budgettet. Dette betyder dog ikke at det ikke kan få store konsekvenser for kunderne af systemet. Konsekvensen er derfor medium: 4.

\subsubsection{Risikoforebyggelse}
Ledelse bør medregne produktivitetsfaldet til at starte med, da det er højst usandsynligt, at det bare giver afkast med det samme. Desuden bør der kigges meget på, hvordan og hvor meget der skal undervises i systemet, da det kan være essentielt for hvor godt medarbejderne kommer igang. Målingerne af systemet bør heller ikke foretages med det samme, men i stedet senere i forløbet og over flere gange, så det er muligt at se forbedringerne. 

\subsubsection{Konsekvensforebyggelse}
Konsekvensen kan forebygges, hvis man sørger for at have ekstra personale klar, som har været med i undervisningen. Der er måske tale om at man har uddannet nogle ekstra vikarer i systemet, som kan træde til, hvis det viser sig at de fastansatte ikke kan holde den forventede kadence.


\subsection{Ansatte flytter ikke med til Hobro}
%Caro
\subsubsection{Årsag}
Arbejdet skal fremover samles i Hobro. Det er ikke sikkert at de ansatte som er uddannet i det gamle system, og bor rundt omkring i hele landet ønsker at flytte til Hobro.

\subsubsection{Konsekvens}
Der skal uddannes nyt personale i selve tinglysningsopgaven, samtidig med at man skal uddanne det gamle personale i det nye system.

\subsubsection{Sandsynlighed}
Sandsynligheden er ret stor for at ikke alle vælger at flytte med: 90\%.

\subsubsection{Konsekvens størrelse}
Ibrugtagningsperioden vil blive forlænget idet man nu udover at skulle indkøre det nye system, også skal indkøre nye medarbejdere i selve tinglysningen. Konsekvens score er 5.

\subsubsection{Risikoforebyggelse}
Risikoen kan forebygges ved at give det eksisterende personale gode grunde til at flytte til Hobro. Højere løn, hjælp med at finde bolig etc. Risikoen kan også forebygges ved tidligt at undersøge hvormange der gerne vil flytte, og dermed tidligt kunne nå at uddanne nye medarbejdere i selve tinglysningen, så de kan uddannes i det nye system samtidig med de "`gamle"' medarbejdere. Sandsynligheden for at ikke alle flytter med vil nok stadig være den samme, men konsekvens scoren vil falde til 3.

\subsubsection{Konsekvensforebyggelse}
Hvis der er ikke er nok personer der har lyst til at flytte til Hobro, er det klart at man hurtigt må sørge for at uddanne nyt personale, så de er klar til at overtage.

\subsection{Nøglepersoner forlader projektet}
%Caro
\subsubsection{Årsag}
En vigtig del af projektgruppen kan forlade projektet grundet et bedre jobtilbud, eller lysten til at lave noget andet.

\subsubsection{Konsekvens}
Hvis det er en af nøglepersonerne kan den del, som personen stod for, nemt komme til at hænge, da det kan være svært at overtage. 

\subsubsection{Sandsynlighed}
Sandsynligheden for dette er omkring 40\%. Da Tinglysningsprojektet er et meget stort og langt projekt, så kan det være svært ikke at få lyst til nye udfordringer undervejs.

\subsubsection{Konsekvens størrelse}
Konsekvensen kan svinge rigtig meget, da det afhænger af hvilke kontraktlige forpligtelser man har lavet til at starte med. Er der f.eks. to måneders opsigelse (i modsætning til normalt en), så personen bedre kan nå at oplære en ny? Kan man ved at tilbyde mere i løn eller lignende få personen til at blive? Hvis personen forlader projektet er konsekvensen nok 4.

\subsubsection{Risikoforebyggelse}
Som tidligere nævnt kan denne risiko forebygges ved at sørge for at give nøglepersoner spændende arbejde og en god løn. Dette alene kan dog ikke hindre en person i at finde et tilbud mere interessant, men konsekvensen kan i det tilfælde mindskes ved at sørge for at opsigelsesfristen er lang nok til at personen vil kunne nå at sætte en ny medarbejder ordentligt ind i arbejdet.

\subsubsection{Konsekvensforebyggelse}
Konsekvensen omkring at projektet kommer til at lide kan mindskes ved at man ansætter en anden person hurtigst muligt, og gerne finder en meget kompetent person, der tidligere har arbejdet med lignende opgaver. Denne person vil hurtigere kunne sætte sig ind i arbejdsgange og forventninger.


\subsection{Leverandør udfører ikke arbejdet hurtigt nok}
%Rene
\subsubsection{Årsag}
Tidsplanen er ikke veldefineret og kan derfor indeholde overraskelser der gør at projektet bliver udskudt.

\subsubsection{Konsekvens}
Projektet vil komme til at koste flere penge, da det enten kræver at det gamle system holdes i live i længere tid eller at der skal sættes flere mand på projektet hos leverandøren. Sidstnævnte er ikke en garanti for, at projektet bliver gennemført hurtigere.

\subsubsection{Sandsynlighed}
Da tidsplanen er så vagt formuleret er sandsynligheden for udskydelse ret stor: 70\%.

\subsubsection{Konsekvens størrelse}
Konsekvensen vil ikke være stor for arbejdsgangen, men vil kun kunne mærkes på pengepungen. Derfor: 6.

\subsubsection{Risikoforebyggelse}
Ved at få leverandøren til at komme med en detaljeret tidsplan for hvor meget tid det tager at gennemføre enkelte delelementer af systemet er det muligt at få et større overblik over, hvad der skal ske.
Den mere detaljerede tidsplan kan også varsle, at projektet måske tager længere tid end forventet allerede fra start og derfor kan budgettet justeres, så der ikke kommer store overraskelser.
Prisen for dette er ikke højere end, at det vil kunne betale sig i den lange ende.

\subsubsection{Konsekvensforebyggelse}
Der kan enten laves en aftale om, at leverandøren betaler en procentdel af de timer, som der arbejdes udover det estimerede, så prisen på projektet ikke kommer til at stige voldsomt.
Eller man kan lave en fast prisaftale. Det er dog ikke sikkert, at det er anbefalelsesværdigt, da det kan give sænket motivation hos leverandøren, hvis de står til at miste penge på projektet.


\subsection{Kunde kommer med for mange ændringer}
%Caro
\subsubsection{Årsag}
Kunden kan have haft problemer med at definere programmet til at starte med. Hvad er det man gerne vil have, og hvordan vil man gerne have det? Det er en stor risiko, der helt sikkert vil sikre øget pris og forsinkelse hvis kunden hele tiden ændrer holdninger og meninger, og ændrer i design etc.

\subsubsection{Konsekvens}
Projektet bliver højst sandsynligt forsinket og sikkert også dyrere, fordi det vil tage leverandøren længere tid at skulle tilbage og ændre allerede færdigkodede ting. Afhængig af størrelsen af ændringer kan det være nødvendigt at skrotte alt det man allerede havde fået lavet og starte forfra.

\subsubsection{Sandsynlighed}
Sandsynligheden er ret stor når der er tale om et stort projekt som Tinglysningsprojektet. Det er meget sandsynligt at der ikke er styr på hver eneste lille detalje omkring et så omfattende stykke programmel. Vi vil mene at risikoen er omkring 70\%.

\subsubsection{Konsekvens størrelse}
Det vil have ret høj konsekvens, da projektet dermed vil blive begravet i ændringer, der kan gøre at evt. næsten færdige dele af programmet skal ændes fundamentalt. Konsekvensen vil dermed være 7.

\subsubsection{Risikoforebyggelse}
Denne risiko kan forebygges ved at der laves et stort stykke indledende designarbejde. Der skal laves et mockup af hele systemet, og det skal designes igennem og testes af brugere inden systemet overhovedet bliver forsøgt implementeret. Dette vil kunne sikre at der kommer problemer senere hen. Samtidig skal leverandøren sørge for at belyse de problempunkter, som de finder i oplægget. Det er op til dem at være opmærksom på de dele der måske ikke er så veldefinerede fra start af, og sørge for at snakke tingene igennem undervejs.

\subsubsection{Konsekvensforebyggelse}
Hvis vi først er i en situation hvor kunder kræver for mange rettelser, kan det være vigtigt at gå tilbage til grundlæggende arbejde. Kigge det igennem, og finde ud af hvad det er der skal ændres. Det er vigtigt at finde ud af hvor det er i forarbejdet det er gået galt, og få rettet det, sådan at der ikke igen kommer problemer.

\section{Risiko analyse i skema}
\begin{tabular}{l|l|l|l}
           & 0-100\%       & 0-10       & P*I   \\
Risiko     & Sandsynlighed & Konsekvens & Score \\\hline
Idriftsættelse & 70\% & 9 & 6,3 \\
Effektivitet   & 90\% & 4 & 3,6 \\
Flytning   & 90\% & 5 & 4,5 \\
Nøglepersoner & 40\% & 4 & 1,6 \\
Leverandør & 70\% & 6 & 4,2\\
Ændringsforslag & 70\% & 7 & 4,9 \\
\end{tabular}


\end{document}
