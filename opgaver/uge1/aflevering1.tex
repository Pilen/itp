\documentclass[10pt,a4paper,danish]{article}
\usepackage[danish]{babel}
\usepackage[utf8]{inputenc}
\usepackage{amsmath}
\usepackage{amssymb}
\usepackage{listings}
\usepackage{fancyhdr}
\usepackage[hidelinks]{hyperref}
\usepackage{booktabs}
\usepackage{graphicx}
\usepackage{xfrac}
\usepackage[dot, autosize, outputdir="dotgraphs/"]{dot2texi}
\usepackage{tikz}
\usepackage{ulem}
\usetikzlibrary{shapes}

\pagestyle{fancy}
\fancyhead{}
\fancyfoot{}
\rhead{\today}
\rfoot{\thepage}
\setlength\parskip{1em}
\setlength\parindent{1em}

%% Titel og forfatter
\title{}
\author{Søren Pilgård, 190689, vpb984\\
Caroline Miller, 04071979, twq135\\
Rene}

%% Start dokumentet
\begin{document}

%% Vis titel
\maketitle
\newpage

%% Vis indholdsfortegnelse
\tableofcontents
\newpage

\section{De fem succeskriterier}

\begin{itemize}
\item Et år efter indførsel skal den gennemsnitlige tid fra en sag er startet til den er afsluttet være reduceret med 20\%
\item Et år efter indførsel skal de personer der dagligt arbejder med systemet svare at konkrete problemer med det gamle system er blevet løst i 75\% af tilfældene
\item Et år efter indførsel skal systemet være integreret med 50\% af andre løsninger i staten
\item Et år efter indførsel skal det være muligt at sikre besparelser på ca. 88,3 mio. kr
\item Et år efter indførsel skal den enkelte ansatte kunne klare 25\% flere sager end i det gamle system
\end{itemize}

\section{Argumentation}



\section{Sikring af realisme og accept}


\end{document}
