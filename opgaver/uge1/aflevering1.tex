\documentclass[10pt,a4paper,danish]{article}
\usepackage[danish]{babel}
\usepackage[utf8]{inputenc}
\usepackage{amsmath}
\usepackage{amssymb}
\usepackage{listings}
\usepackage{fancyhdr}
\usepackage[hidelinks]{hyperref}
\usepackage{booktabs}
\usepackage{graphicx}
\usepackage{xfrac}
\usepackage[dot, autosize, outputdir="dotgraphs/"]{dot2texi}
\usepackage{tikz}
\usepackage{ulem}
\usetikzlibrary{shapes}

\pagestyle{fancy}
\fancyhead{}
\fancyfoot{}
\rhead{\today}
\rfoot{\thepage}
\setlength\parskip{1em}
\setlength\parindent{1em}

%% Titel og forfatter
\title{}
\author{Søren Pilgård, 190689, vpb984\\
Caroline Miller, 04071979, twq135\\
Rene, 070192, vlx198}

%% Start dokumentet
\begin{document}

%% Vis titel
\maketitle
\newpage

%% Vis indholdsfortegnelse
\tableofcontents
\newpage

\section{De fem succeskriterier}

\begin{itemize}
\item Et år efter indførsel skal den gennemsnitlige tid fra en sag er startet til den er afsluttet være reduceret med 20\%
\item Et år efter indførsel skal de personer der dagligt arbejder med systemet svare at konkrete problemer med det gamle system er blevet løst i 75\% af tilfældene
\item Et år efter indførsel skal systemet være integreret med 50\% af andre løsninger i staten
\item Et år efter indførsel skal det være muligt at sikre besparelser på ca. 88,3 mio. kr
\item Et år efter indførsel skal den enkelte ansatte kunne klare 25\% flere sager end i det gamle system
\end{itemize}

\section{Argumentation}
Alle kriterierne som er opstillet ovenfor indeholder målelige krav. Dette gør de fordi de nævner procentsatser og pengebeløb, hvilket gør kravene målbare.

Alle kriterierne skal først indløses efter 1 år, da det kan være svært at måle succes før at systemet er i brug, og brugerne har lært det nye system at kende. I starten vil et nyt system ofte udfordre brugerne, fordi der er en vis læringskurve, og der vil systemet måske virke dårligere end beregnet, men forhåbentligt vil det virke bedre senere.

Det vigtige forretningsmæssige udbytte for Tinglysningsprojektet er at få et system, der virker og som kan samle tinglysningen et sted og digitalt. Derudover bør samlingen af tinglysningen i et centralt system spare penge på medarbejdetimer, og det bør kunne måles om den besparelse holder stik, ved at se på personaleudgifter.

Sager bør kunne klares hurtigere og nemmere i det nye system, og det burde nemt kunne undersøges om der bliver udført 25\% flere sager af hver enkelt ansat, og om hver enkelt sag også tager kortere tid at udføre.

Selvfølgelig er staten også interesserede i at deres investering i et nyt tinglysningssystem vil spare dem penge i det lange løb. Det er klart at det vil være en stor investering i begyndelsen, men hvis systemet kan spare dem for 88,3 mio.kr. om året i personale- og driftsudgifter, så er det investeringen værd.


\section{Sikring af realisme og accept}
Ved starten af projektet, n�r id�en er opst�et, s� skal den skitseres. Det kan g�res i et samarbejde mellem forretnings- og it-udviklingsenheder. N�r skitseringen er f�rdiggjort kan man s�tte den forbi en projektprogramkomit�, der vurderer hvorvidt projektet er egnet til at blive udf�rt. Hvis projektet viser sig at v�re egnet, s� b�r der laves en foranalyse af projektlederen, der munder ud i en projektdefinition, der bl.a. indeholder en business case over projektet. Projektdefinitionen skal s� bed�mmes af interessenterne i systemet, nemlig projektejer, projektsponsor og projektprogramkomit�, der foretager prioritering samt fastl�gger projektets rammer. N�r dette er gjort neds�ttes en styregruppe og projektlederen bemander og starter projektet. Ved at f�lge denne process burde det sikre, at alle interessenter bliver h�rt og derved kan komme med mulige �ndringer til projektet. N�r alle gives en mulighed for at komme med �ndringer til projektet, s� burde det v�re s�dan, at succeskriterierne er realistiske og generelt accepterede.


\end{document}
