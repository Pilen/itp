\documentclass[10pt,a4paper,danish]{article}
\usepackage[danish]{babel}
\usepackage[utf8]{inputenc}
\usepackage{amsmath}
\usepackage{amssymb}
\usepackage{listings}
\usepackage{fancyhdr}
\usepackage{hyperref}
\usepackage{booktabs}
\usepackage{graphicx}
\usepackage{xfrac}
\usepackage[dot, autosize, outputdir="dotgraphs/"]{dot2texi}
\usepackage{tikz}
\usetikzlibrary{shapes}

\pagestyle{fancy}
\fancyhead{}
\fancyfoot{}
\rhead{\today}
\rfoot{\thepage}
\setlength\parskip{1em}
\setlength\parindent{1em}

%% Titel og forfatter
\title{Ugeopgave 5: Systemudviklingsmetode}
\author{Søren Pilgård, 190689, vpb984\\
Caroline Miller, 04071979, twq135\\
René Løwe Jacobsen, 070192, vlx198\\
Paw Saabye Petersen, 110788, vwn245}

%% Start dokumentet
\begin{document}

%% Vis titel
\maketitle
\newpage

%% Vis indholdsfortegnelse
%\tableofcontents
%\newpage

\section{Indledning}




\section{Konklusion}


\begin{thebibliography}{9}

\bibitem{Rigs}
  Rigsrevisionen,
  \emph{Beretning til Statsrevisorerne om det digitale
  tinglysningsprojekt}
  August, 2010

\bibitem{Krav}
  Devoteam Consulting
  \emph{Bilag 2: Kravspecifikation for det kommende
  system til elektronisk tinglysning}
  26. juni 2006

\bibitem{Tid}
  \emph{Bilag 1: Hovedtidsplan}
  26. juni 2006

\end{thebibliography}


\end{document}
