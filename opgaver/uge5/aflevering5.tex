\documentclass[10pt,a4paper,danish]{article}
\usepackage[danish]{babel}
\usepackage[utf8]{inputenc}
\usepackage{amsmath}
\usepackage{amssymb}
\usepackage{listings}
\usepackage{fancyhdr}
\usepackage{hyperref}
\usepackage{booktabs}
\usepackage{graphicx}
\usepackage{xfrac}
\usepackage[dot, autosize, outputdir="dotgraphs/"]{dot2texi}
\usepackage{tikz}
\usetikzlibrary{shapes}

\pagestyle{fancy}
\fancyhead{}
\fancyfoot{}
\rhead{\today}
\rfoot{\thepage}
\setlength\parskip{1em}
\setlength\parindent{1em}

%% Titel og forfatter
\title{Ugeopgave 5: Systemudviklingsmetode}
\author{Søren Pilgård, 190689, vpb984\\
Caroline Miller, 04071979, twq135\\
René Løwe Jacobsen, 070192, vlx198\\
Paw Saabye Petersen, 110788, vwn245}

%% Start dokumentet
\begin{document}

%% Vis titel
\maketitle
\newpage

%% Vis indholdsfortegnelse
%\tableofcontents
%\newpage

\section{Indledning}



\section{Domstyrelsens krav og forventninger til systemudviklingsmetoden}
Domstolsstyrelsen specificerer ikke direkte i deres kontrakt\cite{Kontrakt}
eller Kravspecifikation\cite{Krav} hvilken systemudviklingsmetode der skal
bruges til at udvikle systemet.

Deres tanker fremgår dog gennem Kontraktens bilag 1\cite{Tid}, hvori en tidsplan
er lagt.
I hovedtidsplanen ses det hvordan der først skal udarbejdes specifikationer.
Herefter deles systemet op i mindre dele der hver skal designes og implementeres
i hver sine efterfølgende faser. Af kontrakten punkt 10.1
fremgår det at efter hver af disse faser skal delsystemet af prøves og \textit{``Først når overtagelsesprøven for fasen er skriftligt go dkendt af kunden, anses
  fasen for overtaget (overtagelsesdagen), jfr. punkt 11.''}\cite[s.~16]{Kontrakt}.

Dermed ligger domstolsstyrelsen op til at der bruges en klassisk vandfaldsmodel
hvor man først laver en overordnet specifikation og så går man i gang med
arbejdet. Man har dog valgt at ligge design delen for hvert enkelt delsystem ned
i en fase sammen med implementeringen deraf. Det er dog stadig en vandfaldsmodel
hvor strømmen blot bevæger sig igennem flere små vandfald en et stort.


Kontrakten lader med den skarpe tidsplan dermed ikke leverandøren arbejde på
andre måder.

\section{Leverandørens udviklingsmetode}
Da det hverken står i Rigsrevisionens rapport eller nogen dokumenter relevante til Tinglysningsprojektet, hvilken systemudviklingsmetode der er brugt, så vil vi komme med et kvalificeret bud.
Udfra alle dokumenter, der fortæller os om projektet, bliver vi ledt i retningen af, at der er brugt vandfaldsmodellen til projektet.
Det, der leder os i retningen af, at vandfaldsmodellen er den brugte, er bl.a. tidsplanen, der er lagt for projektet, at kravsspecifikationen er meget udtømmende omkring krav til systemet og
at systemet bliver implementeret ved big-bang.


\section{Konklusion}


\begin{thebibliography}{9}

\bibitem{Rigs}
  Rigsrevisionen,
  \emph{Beretning til Statsrevisorerne om det digitale
  tinglysningsprojekt}
  August, 2010

\bibitem{Kontrakt}
Domstolsstyrelsen
  \emph{Kontrakt}
  26. juni 2006

\bibitem{Krav}
  Devoteam Consulting
  \emph{Bilag 2: Kravspecifikation for det kommende
  system til elektronisk tinglysning}
  26. juni 2006

\bibitem{Tid}
  \emph{Bilag 1: Hovedtidsplan}
  26. juni 2006

\end{thebibliography}


\end{document}
